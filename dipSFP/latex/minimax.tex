\HeaderA{minimax}{Minimax function}{minimax}
\begin{Description}\relax
Determine if each point in a line is a local maximum or minimum.
\end{Description}
\begin{Usage}
\begin{verbatim}
minimax(func)
\end{verbatim}
\end{Usage}
\begin{Arguments}
\begin{ldescription}
\item[\code{func}] A matrix. Each row indicates the (x,y)-coordinate of one point.

\end{ldescription}
\end{Arguments}
\begin{Value}
A vector with the same length as the first dimension of matrix
\code{func}, composed of -1, 1, and 0, indicating respectively
the corresponding point is a local minimum, maximum or none of them.
\end{Value}
\begin{Author}\relax
Na You
\end{Author}
\begin{References}\relax
Fisher, N. I. and Marron, J. S. (2001).
Mode testing via the excess mass estimate. Biometrika, 88(2):499--517.
\end{References}
\begin{Examples}
\begin{ExampleCode}
x <- rnorm(10,0,1)
minimax(cbind(seq(x),x))
\end{ExampleCode}
\end{Examples}


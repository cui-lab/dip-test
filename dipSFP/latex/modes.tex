\HeaderA{modes}{Modes of density function}{modes}
\begin{Description}\relax
Given a vector of data, find the `major' modes of its Gaussian kernal
density estimate with specified bandwidth, exluding the modes caused
by outliers and `minor' modes with small masses.
\end{Description}
\begin{Usage}
\begin{verbatim}
modes(vec, bw, lambda0 = NULL, outliers.ratio = NULL, m0 = NULL)
\end{verbatim}
\end{Usage}
\begin{Arguments}
\begin{ldescription}
\item[\code{vec}] a vector.

\item[\code{bw}] bandwidth used in the Gaussion kernal density estimate.

\item[\code{lambda0, outliers.ratio, m0}] tuning parameters. See `Detailes' in \code{FMbw}.

\end{ldescription}
\end{Arguments}
\begin{Value}
A vector, each element of which indicates the location of one mode.
\end{Value}
\begin{Author}\relax
Na You
\end{Author}
\begin{References}\relax
Fisher, N. I. and Marron, J. S. (2001).
Mode testing via the excess mass estimate. Biometrika, 88(2):499--517.
\end{References}
\begin{SeeAlso}\relax
\code{\LinkA{FMbw}{FMbw}}
\end{SeeAlso}
\begin{Examples}
\begin{ExampleCode}
x <- rnorm(100,0,1)
h1 <- FMbw(x,k=1)
modes(x,bw=h1)

h2 <- FMbw(x,k=2)
modes(x,bw=h2)
\end{ExampleCode}
\end{Examples}


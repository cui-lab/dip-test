\HeaderA{FMbw}{Modified critical bandwidth}{FMbw}
\begin{Description}\relax
Modified critical bandwidth in Fisher and Marron (2001).
\end{Description}
\begin{Usage}
\begin{verbatim}
FMbw(vec, k, lambda0 = NULL, outliers.ratio = NULL, m0 = NULL, tol = 1e-06)
\end{verbatim}
\end{Usage}
\begin{Arguments}
\begin{ldescription}
\item[\code{vec}] a vector.

\item[\code{k}] the number of modes.

\item[\code{lambda0,outliers.ratio,m0}] tuning parameters, see `Details' below.

\item[\code{tol}] the desired accuracy (convergence tolerance) in computing the
modified critical bandwidth by bisection method.

\end{ldescription}
\end{Arguments}
\begin{Details}\relax
Fisher and Marron (2001) proposed a modified critical bandwidth
instead of Silverman critical bandwidth (1981) to estimate the
underlying data distribution and generate bootstrap resamples in
smooth bootstrap algorithm. Two tuning parameters `lambda0' and `m0'
were used to deal with the outliers problems and distinguish minor or
major modes. As suggested in Fisher and Marron (2001),
`lambda0' can be given directly by
user or determined by the ratio of outliers `outliers.ratio'.
If both of them are given, then the maximum of `lambda0' and
`outliers.ratio/h/(2*pi)\textasciicircum{}(1/2)' is used in computation, where `h' is
the bandwidth at which the statistic S\_k is calculated. The defaults
of `lambda0' and `outliers.ratio' are both zero. The default of `m\_0'
is 0.003.
\end{Details}
\begin{Value}
The value of modified critical bandwidth.
\end{Value}
\begin{Author}\relax
Na You
\end{Author}
\begin{References}\relax
Fisher, N. I. and Marron, J. S. (2001).
Mode testing via the excess mass estimate. Biometrika, 88(2):499--517.
\end{References}
\begin{Examples}
\begin{ExampleCode}
x <- rnorm(100,0,1)
FMbw(x,1)
FMbw(x,1,m0=0)
\end{ExampleCode}
\end{Examples}


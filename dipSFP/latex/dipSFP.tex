\HeaderA{dipSFP}{SFP detection procedure based on dip statistic}{dipSFP}
\begin{Description}\relax
SFP detection procedure based on dip statistic, where the null
distribution of dip statistic is estimated by smooth bootstrap algorithm.
\end{Description}
\begin{Usage}
\begin{verbatim}
dipSFP(dat, nboot, lambda0 = NULL, outliers.ratio = NULL, m0 = NULL,
pvalue = TRUE, tol = 1e-06)
\end{verbatim}
\end{Usage}
\begin{Arguments}
\begin{ldescription}
\item[\code{dat}] a matrix of microarray data to be analyzed, for example,
the estimated binding affynity values or SFPdev values, where each row
corresponds to one probe and each column corresponds to one chip.

\item[\code{nboot}] the number of bootstrap replications.

\item[\code{lambda0,outliers.ratio,m0}] tuning parameters, see `Details' in \code{FMbw}.
\item[\code{pvalue}] logical; will p values be computed? 

\item[\code{tol}] the desired accuracy (convergence tolerance) in computing the
modified critical bandwidth by bisection method.

\end{ldescription}
\end{Arguments}
\begin{Details}\relax
Dip statistic is computed by \code{dip} in R package `diptest'.
\end{Details}
\begin{Value}
A list with four components if \code{pvalue} is TRUE,

\begin{ldescription}
\item[\code{dip.statistics}] dip statistic values for probes
\item[\code{null.distribution}] estimated null distribution of dip statistic
\item[\code{p.values}] p values
\item[\code{adjusted.pvalues}] adjusted p values using the false discovery
rate criterion
\end{ldescription}


If \code{pvalue} is FALSE, only \code{dip.statistics} and
\code{null.distribution} are returned.
\end{Value}
\begin{Author}\relax
Na You
\end{Author}
\begin{References}\relax
Cui et al.
Single Feature Polymorhpism Detection using Recombinant Inbred Line
microarray expression data.

Fisher, N. I. and Marron, J. S. (2001).
Mode testing via the excess mass estimate. Biometrika, 88(2):499--517.

Hartigan, J. A. and Hartigan, P. M. (1985).
The dip test of unimodality. The Annals of Statistics, 13(1):70–84.
\end{References}
\begin{SeeAlso}\relax
\code{\LinkA{FMbw}{FMbw}}
\end{SeeAlso}
\begin{Examples}
\begin{ExampleCode}
n <- 100
m.nonSFP <- 98
m.SFP <- 2

dat1 <- matrix(rgamma(m.nonSFP*n,2,4),m.nonSFP,n)
dat2 <- cbind(matrix(rgamma(m.SFP*n/2,2,4),m.SFP,n/2),
              matrix(rgamma(m.SFP*n/2,20,4),m.SFP,n/2))
dat <- rbind(dat1,dat2)

dipSFP(dat,nboot=20)
dipSFP(dat,nboot=20,outliers.ratio=0.01)
\end{ExampleCode}
\end{Examples}

